\documentclass[preprint, 3p,
authoryear]{elsarticle} %review=doublespace preprint=single 5p=2 column
%%% Begin My package additions %%%%%%%%%%%%%%%%%%%

\usepackage[hyphens]{url}

  \journal{Journal of Transport Geography} % Sets Journal name

\usepackage{graphicx}
%%%%%%%%%%%%%%%% end my additions to header

\usepackage[T1]{fontenc}
\usepackage{lmodern}
\usepackage{amssymb,amsmath}
% TODO: Currently lineno needs to be loaded after amsmath because of conflict
% https://github.com/latex-lineno/lineno/issues/5
\usepackage{lineno} % add
\usepackage{ifxetex,ifluatex}
\usepackage{fixltx2e} % provides \textsubscript
% use upquote if available, for straight quotes in verbatim environments
\IfFileExists{upquote.sty}{\usepackage{upquote}}{}
\ifnum 0\ifxetex 1\fi\ifluatex 1\fi=0 % if pdftex
  \usepackage[utf8]{inputenc}
\else % if luatex or xelatex
  \usepackage{fontspec}
  \ifxetex
    \usepackage{xltxtra,xunicode}
  \fi
  \defaultfontfeatures{Mapping=tex-text,Scale=MatchLowercase}
  \newcommand{\euro}{€}
\fi
% use microtype if available
\IfFileExists{microtype.sty}{\usepackage{microtype}}{}
\usepackage[]{natbib}
\bibliographystyle{elsarticle-harv}

\ifxetex
  \usepackage[setpagesize=false, % page size defined by xetex
              unicode=false, % unicode breaks when used with xetex
              xetex]{hyperref}
\else
  \usepackage[unicode=true]{hyperref}
\fi
\hypersetup{breaklinks=true,
            bookmarks=true,
            pdfauthor={},
            pdftitle={A historical analysis of the evolution of active travel behaviour in Canada},
            colorlinks=false,
            urlcolor=blue,
            linkcolor=magenta,
            pdfborder={0 0 0}}

\setcounter{secnumdepth}{5}
% Pandoc toggle for numbering sections (defaults to be off)


% tightlist command for lists without linebreak
\providecommand{\tightlist}{%
  \setlength{\itemsep}{0pt}\setlength{\parskip}{0pt}}




\usepackage{pdflscape}
\usepackage{booktabs}
\usepackage{longtable}
\usepackage{array}
\usepackage{multirow}
\usepackage{wrapfig}
\usepackage{float}
\usepackage{colortbl}
\usepackage{pdflscape}
\usepackage{tabu}
\usepackage{threeparttable}
\usepackage{threeparttablex}
\usepackage[normalem]{ulem}
\usepackage{makecell}
\usepackage{xcolor}



\begin{document}


\begin{frontmatter}

  \title{A historical analysis of the evolution of active travel
behaviour in Canada}
    \author[Some Institute of Technology]{Alice Anonymous%
  \corref{cor1}%
  }
   \ead{alice@example.com} 
    \author[Some Institute of Technology]{Bob Security%
  %
  }
   \ead{bob@example.com} 
    \author[Some Institute of Technology]{Cat Memes%
  %
  }
   \ead{cat@example.com} 
      \affiliation[Some Institute of Technology]{
    organization={Big Wig University},addressline={1 main
street},city={Gotham},postcode={123456},state={State},country={United
States},}
    \cortext[cor1]{Corresponding author}
  
  \begin{abstract}
  Active transportation (AT), defined as self-powered modes such as
  walking and cycling, can help individuals meet health recommendations
  of 150 minutes of moderate-to-vigorous physical activity per week.
  Despite the potential of Canada's Time Use Survey (TUS) from the
  General Social Survey (GSS) to inform AT research, no comprehensive
  historical analysis of AT using all TUS cycles has yet been conducted.
  This study addresses that gap with two objectives: to examine temporal
  trends in AT by destination, travel time, and demographic profiles;
  and to calibrate impedance functions for AT modes across survey cycles
  and destinations. After analyzing and processing over 13,500 AT
  records representing 28 million weighted episodes, we performed
  descriptive analyses and modeled a wide range of impedance functions.
  Results show that ``Home,'' ``Work or study,'' and ``Grocery store''
  were the most frequent destinations. Walking dominated AT (over 90\%
  of episodes), with median durations rising from 10 to 15 minutes in
  2022; for cycling, durations rose from 15 to 30 minutes. Since 2010,
  the share of individuals with AT episodes declined, especially among
  Women+, reversing a previous gender pattern. The group cohort between
  15 to 24 years remained as the most active, while adults older than 75
  years showed steady increases. All fitted impedance functions deviated
  from exponential form, indicating that standard assumptions about
  patterns of distance decay functions may misrepresent AT behaviour,
  particularly for short trips. These findings improve our understanding
  of active travel trends and provide empirical support for AT
  accessibility measures and transportation policy.
  \end{abstract}
    \begin{keyword}
    Active mobility \sep Walking \sep Cycling \sep Impedance
function \sep 
    Temporal evolution
  \end{keyword}
  
 \end{frontmatter}

\section{Introduction}\label{introduction}

The Canadian Society for Exercise Physiology (CSEP) recommends that
adults aged 18 to 64 accumulate at least 150 minutes of moderate-to
vigorous-intensity aerobic physical activity per week, in bouts of 10
minutes or more \citep{csep2012}. Moderate-intensity activities are
those that typically cause adults to sweat slightly and breathe harder,
such as brisk walking and bicycling. In contrast, vigorous-intensity
activities cause individuals to sweat more heavily and become out of
breath, including activities like running, basketball, soccer, and
cross-country skiing. The health benefits of achieving the recommended
150 minutes per week (approximately 21 minutes per day) include a
reduced risk of premature death \citep{hakim1998effects}, heart disease
\citep{lacroix1996, hakim1999}, stroke \citep{hu2000}, high blood
pressure \citep{dunn1999}, certain cancers, type 2 diabetes
\citep{hu1999}, osteoporosis, overweight, and obesity
\citep{fogelholm2000}. Regular physical activity also contributes to
improved fitness, strength, and mental health, including better morale
and self-esteem \citep{csep2012}.

Active transportation (AT) is an important source of moderate-to
vigorous-intensity physical activity and can help individuals meet
recommended activity levels \citep{bryan2009}. AT refers to
non-motorized and self-powered forms of travel, including walking,
cycling, and the use of aids such as wheelchairs, scooters, e-bikes,
rollerblades, snowshoes, and cross-country skis \citep{csep2012}.
Walking, in particular, is of interest for promoting physical activity
among inactive populations because it is low-cost, easily integrated
into daily routines, requires no special equipment or training, and
carries a relatively low risk of injury compared to more intense
physical activities \citep{hootman2001, bryan2009}. In turn, cycling
provides greater health benefits compared to walking, due to the greater
intensity and duration associated with this mode
\citep{martin2014, barajas2021, borhani2024, celismorales2017}.

Additionally, both walking and cycling play a important role in
enhancing and promoting urban sustainability
\citep{hino2014built, lamiquiz2015effects}, making them central to urban
mobility research and policy-making
\citep{vandenbulcke2009mapping, wu2019measuring}. Walking and cycling
accessibility, the ease of reaching destinations and opportunities
\citep{hansen1959, paez2012} by walking and cycling, are closely related
and together contribute to the concept of ``active accessibility'' or
``non-motorized accessibility.'' When incorporated into urban and
transportation planning, they help reduce dependence on private vehicles
and promote healthier, more sustainable travel behaviour among
residents.

In 2021, Canada released the ``National Active Transportation Strategy''
\citep{canada2021} to support the expansion and enhancement of active
transportation infrastructure in Canada. The federal government
committed \$400 million over five years to build and improve networks of
pathways, bike lanes, trails, and pedestrian bridges. Beyond its
well-established health benefits, the strategy outlines several
additional advantages of expanding AT, including: economic benefits,
such as savings on household transportation costs (e.g., fewer
vehicle-related expenses, trips, and parking needs), increased tourism
and the growth of outdoor and eco-tourism, and increased foot traffic
and spending at businesses accessible via AT; environmental benefits,
such as improved air quality and environmental resilience due to a
higher modal share of AT, reduced land consumption for roads and
parking, and decreased water pollution from runoff due to paved
surfaces; in addition to social benefits, such as increased public space
for social interaction, and improved access to amenities, health care,
education, and social services.

In Canada, the Time Use Survey (TUS) cycles of the General Social Survey
(GSS), administered by Statistics Canada, offer valuable data for
analyzing Canadians' travel behaviour \citep{statisticscanada2022}. This
diary-based survey records individuals' activities over 24 hours to
capture societal changes related to living conditions and well-being.
TUS cycles have been conducted every five to seven years since 1986.
Respondents report their main and simultaneous activities, their
duration, location, if other persons accompany them, and, more recently,
whether information technology was used during the activity.

The TUS allows researchers to identify the origin and destination of
trips, travel times, and transportation modes used, providing a valuable
dataset for analyzing active travel behaviour. It also offers the
empirical foundation for tools used in transportation analysis, such as
the development of impedance functions for accessibility analysis.
Despite this potential, to our knowledge, no prior studies have
conducted a historical analysis of active travel behaviour in Canada
using the full set of TUS GSS cycles.

In transportation research, the travel behaviour of a population can be
represented using impedance functions. These functions can take many
forms but all serve as tools to understand travel behaviour, as they
measure the willingness to travel a certain distance to reach a desired
destination where a service or opportunity is located
\citep{papa2012gravity, yang2012walking, millward2013active, vale2017influence}.
For instance, they help us understand how far people typically walk or
cycle when travelling from an origin to a destination - an important
piece of information for urban planners when allocating services and
opportunities. However, to accurately represent people's willingness to
travel, these functions must be calibrated using real travel behaviour
data. The lack of calibrated impedance functions, especially for
destinations beyond workplaces, poses a challenge for incorporating
robust accessibility indicators into urban and transportation planning
\citep{pereira2023}. Fortunately, in the case of Canada, the TUS
provides behavioural data that enable the calibration of such functions.

Given this context, this study poses the following research questions:
How can active trips in Canada be characterized in terms of their main
origins, destinations, and typical durations? How is the population
engaged in AT distributed across gender and age cohorts? Have these
episodes and the active population varied over the last decades?
Finally, how can these travel behaviours be represented using impedance
functions? To answer these questions, this study has two main
objectives: to investigate the Canadian TUS GSS cycles from 1992 to
2022, providing an overview of AT in terms of main origins,
destinations, travel times, and the demographic profile of individuals
engaging in AT; and to calibrate impedance functions that represent
Canadians' travel behaviours for walking and cycling, considering a wide
range of destinations and time periods across Canadian metropolitan
areas. To achieve both objectives, we utilize data provided by the
\texttt{ActiveCA} R package \citep{dossantos2025}, an open data product
in the form of an R data package with information about active travel in
Canada. This data product is based on Public Use Microdata Files of TUS
GSS cycles. To build this package, the authors extracted all walking and
cycling episodes and their corresponding episode weights for GSS cycles,
Cycles 2 (1986), Cycles 7 (1992), 12 (1998), 19 (2005), 24 (2010), 29
(2015), and 34 (2022), spanning a period of almost fourty years. Origins
and destinations were labelled, enabling the investigation of active
travel for broad destination categories and purposes.

Our research advances the current understanding of active travel
behaviour by analyzing the evolution of travel times and trip frequency
for walking and cycling, trends in the prevalence of AT within the
Canadian population, and the development of calibrated impedance
functions tailored to different destinations, periods, and AT modes.
This study encompasses a wide variety of trip purposes, including
commutes to homes, workplaces, or educational institutions; social
visits; outdoor activities; business trips; shopping; cultural outings
to libraries, museums, or theaters; dining out; and participation in
religious practices. Additionally, we also analyze multimodal travel,
evaluating connections between walking and cycling with motorized
vehicles (including cars as drivers or passengers, taxis, vans, and
motorcycles) and public transit (such as buses, trains, subways, and
streetcars).

We ensured the transparency and reproducibility of our study by sourcing
all data from publicly available repositories. To facilitate
collaboration and further analysis, we developed this paper using
literate programming, with the data analysis code accessible through our
GitHub page \emph{(link to be provided after review)}, in alignment with
best practices in spatial data science \citep{arribas-bel2021}. These
contributions enhance the understanding of active transportation,
emphasize its role in shaping more sustainable mobility strategies, and
provide a foundation for further research and policymaking.

\section{Theoretical background}\label{theoretical-background}

\subsection{Evolution of active travel in
Canada}\label{evolution-of-active-travel-in-canada}

When analyzing patterns and trends in walking and cycling behaviour
among Canadian adults on a national scale, two important studies are
particularly prominent. The first, conducted by Bryan et al.
\citeyearpar{bryan2009}, examined walking behaviours among Canadian
adults aged 18 to 55 using nationally representative cross-sectional
data from the National Population Health Survey and the Canadian
Community Health Survey spanning from 1994/95 to 2007. The authors
calculated the weighted and age-standardized prevalence of walking for
exercise, walking duration, regular walking (defined as walking at least
four times per week), and whether 100\% of leisure-time physical
activity energy expenditure (LTPAEE) was derived from walking. Overall,
70\% of Canadian adults reported walking for exercise at least once
during the previous three months; however, only 30\% reported walking
regularly - a figure that remained relatively stable since 2001. Regular
walking was more commonly reported among women, older adults,
individuals with lower body mass index (BMI), and those in lower-income
households. Similarly, women, older adults, and lower-income Canadians
were more likely to derive 100\% of their total LTPAEE from walking
compared to men, younger adults, and those in higher-income groups. The
study concluded that while walking is a widely practiced form of
physical activity across demographic groups, the prevalence of regular
walking varies considerably by age, sex, BMI, and income.

The second and more recent study, conducted by Borhani et al.
\citeyearpar{borhani2024}, explored active transportation (AT) patterns
in Canada using data from four national health surveys: the National
Population Health Survey (1994--1998), the Canadian Community Health
Survey (2000--2020), the Canadian Health Measures Survey (2007--2019),
and the Health Behaviour in School-aged Children Study (2010--2018).
Their analysis assessed the prevalence of AT participation and time
spent on active trips, focusing on walking and cycling, with results
stratified by age group and sex. The authors noted that inconsistencies
in AT survey questions over time and across surveys posed challenges to
interpreting long-term trends. Even so, they found that females
consistently reported higher levels of walking, while males were more
likely to cycle. Regardless of mode, males reported spending more total
time in AT. Participation in AT decreased with age, with the highest
prevalence observed among youth and the longest durations among young
adults.

\subsection{The Time Use Surveys in
Canada}\label{the-time-use-surveys-in-canada}

TUS surveys provide a valuable source of information on the daily
activities of individuals and households. According to Harms et al.
\citeyearpar{harms2018}, more than 65 countries around the world
(including in Europe, the Americas, Asia, Africa, Australia, and New
Zealand) have conducted together over one hundred TUS.

\subsubsection{The Time Use Surveys in
Canada}\label{the-time-use-surveys-in-canada-1}

In Canada, the TUS cycles of the GSS have provided a comprehensive
cross-sectional snapshot of the Canadian population since 1986
\citep{statisticscanada2022}. Until 2022, Statistics Canada used a
telephone-based sampling frame, which was replaced by a dwelling-based
frame in the most recent cycle. Most respondents to the 2022 TUS
answered the survey online. According to Statistics Canada
\citeyearpar{statisticscanada2022}, this new approach reflected the need
to adapt to changes in the use of technology and the increasing time
demands of Canadians, offering respondents greater flexibility and
convenience in completing the survey. However, it is important to note
that significant changes in survey methodology can affect the
comparability of data over time. It is not possible to determine with
certainty whether, or to what extent, the differences observed in the
variables are attributable to actual changes in the population or to
methodological changes in data collection. At all stages of processing,
verification and dissemination, considerable efforts have been made to
produce data with a high level of accuracy and to ensure that the
published estimates meet Statistics Canada's quality standards. Even so,
there is reason to believe that the use of an electronic questionnaire
may have influenced the estimates. The potential impact of the
collection mode was analyzed by Statistics Canada for a selected set of
key questions. Due to the limitations of the sample size, it was not
possible to carry out this analysis for all the variables. It is worth
mentioning that none of the variables used in this research are listed
in Statistics Canada's 2022 Public Use Microdata File User Guide
\citep{statisticscanada2022} as unsuitable for trend analysis due to
differences in data collection mode.

Eligibility for participation requires individuals to be 15 years of age
or older. Each survey cycle spans a full 12-month period, typically from
July to July of the following year. The target population includes all
Canadians aged 15 and older, with the exception of residents of the
Yukon, Northwest Territories, and Nunavut, full-time residents of
institutions, and individuals living on Indigenous reserves. The TUS
covers both rural and urban areas, encompassing metropolitan and
non-metropolitan regions, thereby ensuring a diverse and representative
sample of the Canadian population. For sampling purposes, the ten
provinces of Canada were divided into distinct geographic strata.
Several Census Metropolitan Areas (CMAs) - including St.~John's,
Halifax, Saint John, Montreal, Quebec City, Toronto, Ottawa, Hamilton,
Winnipeg, Regina, Saskatoon, Calgary, Edmonton, and Vancouver - and some
Census Agglomerations (CAs) were treated as separate strata. Additional
strata were created by grouping other CMAs within Quebec, Ontario, and
British Columbia, as well as by categorizing non-CMA areas within each
province into their own distinct strata.

\subsection{Representation of travel behaviour with impedance
functions}\label{representation-of-travel-behaviour-with-impedance-functions}

Accessibility is the main benefit provided by the transportation system
\citep{pereira2017}, being understood as the potential to access
spatially distributed opportunities \citep{hansen1959, paez2012}. When
computing accessibility measure, is necessary take into account the
challenges associated with this access to different locations and
opportunities. Usually, the effect of travel costs is expressed by
``impedance functions'', also called ``distance decay functions''
\citep{soukhov2024}. Overall, impedance functions are derived from
estimates based on distributions of sample data that reflect variations
in the willingness of individuals to travel different distances to reach
opportunities \citep{li2020approach}. Their objective is to describe the
decrease in the intensity of interaction as the cost of travel between
locations increases. The cost of travel is usually measured in terms of
the distance between the places of origin and destination, or in terms
of the time spent reaching the destination from the point of origin.

Examining the impedance functions across different modes of transport
and destinations is a good way to understand the travel behaviour
associated with each mode, while also helping to examine allegations
about travel behaviour. Current interest in creating ``livable''
communities often relies on broad assumptions about individuals'
willingness to walk or bike to different destinations. For example, it
is commonly assumed that people are generally willing to walk up to a
quarter mile to access most places \citep{untermann1984, larsen2010}.
Similarly, the recent ``15-minute city'' concept proposes that the
majority of daily necessities should be accessible by walking or cycling
within 15 minutes \citep{moreno2021}.

Different categories of accessibility measures have been developed, such
as indicators based on actives, infrastructure, individuals and
utilities \citep{geurs2004, paez2012}. The family of gravity-based
accessibility have been widely used in active modes \citep{miller2005}.
Many gravity-based accessibility measures derive from the work of Hansen
\citeyearpar{hansen1959}, represented in (Equation
\ref{eq:accessibility-equation}), in which an impedance function weights
opportunities:

\begin{equation}
A_{i} = \sum_{j=1}^J O_j .f(c_{ij})
\label{eq:accessibility-equation}
\end{equation}

The accessibility score \(A_{i}\) at each origin \(i\) is obtained by
summing up the opportunities \(O\) available at destination \(j\), where
\(i\) and \(j\) are sets of spatial units in a region. However, the
number of opportunities in each destination is gradually discounted as
travel costs become higher and the rate at which this weight decreases
is determined by a decay function. \(f(c_{ij})\) represents the
impedance during the trip from origin \(i\) to destination \(j\) and
\(c_{ij}\) reflects the generalized travel cost, potentially
encompassing factors such as time, distance and effort. In this way, the
impedance function \(f(c_{ij})\) allows the accessibility analyst to
define a measure of travel behaviour with precision: the relationship
between the ``population'' at an origin and where they normally want to
or can go to reach ``opportunities'' at destinations. The definition of
the impedance function \(f(c_{ij})\) is very important from this
perspective \citep{soukhov2024}.

Since the beginning applications of the gravity-accessibility models, a
range of impedance functions have been applied to describe the
distribution of walking and cycling trips, whether for general or
specific purposes
\citep{iacono2008access, iacono2010, larsen2010, yang2012walking, millward2013active, vale2017influence, li2020approach}.
Selecting an appropriate impedance function can be challenging and
results in a diverse range of cost decay functions that are employed as
impedance functions in accessibility measures, including threshold
functions and smooth cost decay functions (e.g., log-normal, normal,
gamma, and exponential function)
\citep{de2009exponential, reggiani2011accessibility, osth2016new}.

Another type of family of accessibility measures are \emph{cumulative
opportunity} metrics, commonly referred to as isochronous indices. The
binary function Equation \ref{eq:cumulative-equation} forms the basis of
the cumulative opportunities measure approach. This function determine
accessibility by summing up the number of opportunities available within
a specific threshold of travel time or distance from a reference point,
without discounting the potential of the trip in relation to the
associated cost. They use a rectangular function, categorizing the trip
as ``acceptable'' within certain limits and ``unacceptable'' beyond
them. One of the main complexities of these metrics is deciding what the
appropriate threshold point is. This decision may be based on the
prevailing mobility patterns of the population or may reflect
established norms, conventions or informed projections of the
researcher. Note that the cumulative opportunity measure can be
understood as a special case of a gravity-based measure in which the
weight of each opportunity is defined by a binary function, rather than
a gradually decaying function \citep{pereira2023}.

\begin{equation}
C_{ij} =
\begin{cases}
  1 & \text{if } c_{ij} \le x \\
  0 & otherwise
\end{cases}
\label{eq:cumulative-equation}
\end{equation}

Among the various mathematical forms that can represent impedance
functions, the negative exponential function is the dominant choice in
accessibility research
\citep{hansen1959, apparicio2008comparing, iacono2008access, larsen2010, millward2013active}.
Its high adoption can be attributed mainly to its ability to give
greater weight to nearby opportunities, and greater weight to distant
opportunities - a highly relevant characteristic for active modes of
transportation, such as walking and cycling. When Hansen
\citeyearpar{hansen1959} introduced their accessibility measure, the
author applied and indicated the use of exponential distributions
\((e ^ {-\beta x})\) as the impedance function. After this, several
other studies
\citep{fotheringham1989spatial, de2009exponential, iacono2010, signorino2011gravity, prins2014many}
use the negative exponential function after comparison with empirical
trip distribution data.

\section{Materials and Methods}\label{materials-and-methods}

To investigate the historical active travel behaviour in Canada, we
analyzed six GSS Time Use cycles: Cycles 7 (1992), 12 (1998), 19 (2005),
24 (2010), 29 (2015) and 37 (2022). We excluded Cycle 2 (1986) from our
analysis because this survey did not specify whether the respondent
lived in a metropolitan area and did not present cycling as a option of
transportation mode, although this cycle is notable for having been the
first national random sample to examine Canadian time-use patterns. This
paper is a direct application of the ready-to-use data set provided by
the \texttt{ActiveCA} data package \citep{dossantos2025}, which is based
on the Main and Episode files from the GSS Public Use Microdata Files.
The Main file contains questionnaire responses and associated data from
participants, while the Episode files provided detailed information
about every activity episode reported by the respondents.

The methodology involves two main steps. The first step employs
descriptive analysis of AT episodes to identify typical travel times
across destinations and years, comparing their temporal evolution and
identifying differences in AT episodes through statistical tests, and
assess the active population in terms of sex and age group. The second
step calculates and analyzes impedance functions for each combination of
cycle, destination, and AT mode.

To facilitate collaboration and further analysis, we updated the
\texttt{ActiveCA} R Package to include the methodology to obtain
impedance functions from the raw data files (GSS surveys). Additionally,
we created this paper using literate programming in which the R markdown
code to fully reproduce this article is available on our GitHub
repository \emph{(include after the review)}, in line with the best
practices of spatial data science \citep{arribas-bel2021, paez2021}.

\subsection{Analyzing active travel
episodes}\label{analyzing-active-travel-episodes}

A TA episode refers to a walking or cycling activity that a person did
the day before the TUS interview. For each selected cycle of the GSS
surveys, we reviewed the episode files to identify cases with activities
listed as walking or cycling, selecting the locations immediately before
and after the mobility episode. With this process, we were able to
identify the origin and the destination of the active travel episode. We
labeled the code variables with their appropriate descriptions,
identifying the transportation mode, activity/reason of the travel, as
well the province and urban classification of the respondent's residency
(if the respondent lives in a CMA or in a Census Agglomerations).

Additionally, it was necessary to guarantee the data consistency across
the surveys, since they have employed a variety of variable coding
schemes. The range of activities and destinations considered in the
surveys changed from 1992 to 2022. In 1992, there were only three
options of origin/destination location available to the respondent:
their home, other's home and work or study. In its turn, the most recent
survey (2022) counts with twelve possible destination, including sport
area (sports centre, field or arena), restaurant (including bar and
club), health clinics (medical, dental or other health clinic), grocery
stores (including other types of stores and malls) and more. In order to
achieve uniformity, the activity categories from 2005, 2010, 2015, and
2022 were synchronised, and a similar process was employed for those
from 1992 and 1998. For the preceding years (1992, and 1998), the trip
origins and destinations were classified as ``Home,'' ``Other's home,''
and ``Work or school.'' In the subsequent years (2005, 2010, 2015, and
2022), these categories were expanded to include ``Business,''
``Restaurant'' ``Place of worship,'' ``Grocery store''
``Neighbourhood,'' ``Outdoors,'' ``Cultural venues'' (such as library,
museum and theatre), and ``Sport area.''

Statistical analysis was used to characterize active travel episodes
using cross-tabulations and graphs. Summary statistics and visualization
techniques, including median values as a measure of typical value and
box plots, were employed to describe active travel across years,
destinations, and transportation modes. To assess the statistical
significance of potential temporal differences in the empirical episode
data set for each destination, we applied the Kruskal-Wallis test - a
test that evaluates differences in the medians of the empirical data.
This test was chosen because it does not assume a normal distribution
for the data, an important consideration since we made no assumptions
about the distribution of the empirical data.

\subsection{Analyzing the population with active travel
records}\label{analyzing-the-population-with-active-travel-records}

After assessing the active travel episodes, we analyzed the population
with records of active travel for each year of the analysis, stratifying
the analysis by gender and age group. To stratify by gender, we adopted
the definition used in the most recent TUS (2022), which was the first
in the series to consider gender - recognizing a broader spectrum of
gender identities - instead of biological sex, which had previously been
limited to female and male categories. In the 2022 survey, due to the
small size of the non-binary population, data aggregation was necessary
to protect the confidentiality of respondents
\citep{statisticscanada2025}. As a result, information from the TUS 2022
is disseminated using a two-category gender variable. In this framework,
individuals identifying as non-binary are distributed across the two
other gender categories and are denoted by the ``+'' symbol. The
category ``Men+'' includes men (and/or boys) as well as some non-binary
persons, and the category women+ includes women (and/or girls) as well
as some non-binary persons.

We recognize that biological sex does not always align with gender
identity. However, to enable comparison across survey years, we assumed
that individuals recorded as ``female'' in earlier surveys correspond to
women+, and those recorded as ``male'' correspond to men+. We
acknowledge that this approach does not fully capture gender diversity,
particularly for trans and non-binary individuals, due to limitations in
how gender was recorded in earlier surveys.

We also stratified the analysis by age group. To ensure the consistency
across all survey years, we defined the following cohorts: ``15 to 24
years'', ``25 to 34 years'', ``35 to 44 years'', ``45 to 54 years'',
``55 to 64 years'', ``65 to 74 years'', and ``75 years and over''.

After defining the gender and age group categories, we identified the
population with and without at least one active travel episode,
considering both modes (walking and cycling). To complete the population
analysis, we measured and examined the temporal evolution of the number
of active trips per person for each survey year, considering both the
active population and the general population. We also analyzed the
number of walking episodes per person who reported walking activity, and
the number of cycling episodes per cyclist, to identify possible trends
within each AT mode. Finally, we compared the average duration of active
travel episodes to observe changes over time.

\subsection{Estimating impedance function
parameters}\label{estimating-impedance-function-parameters}

After analyzing the AT episodes in terms of the population with AT
records, we calculated impedance functions to represent the transport
behaviour of Canadians for each survey year and destination. One can
adopt different forms of impedance functions to model the population's
transport behaviour. In this study, we adopted Probability Density
Functions (PDFs), following the approach of Soukhov and Paez
\citeyearpar{soukhov2024}. With a PDF, \(f()\) can be interpreted as the
probability density of a trip occurring for each value of travel cost
\(c_{ij}\). If a graph of the PDF (y-axis) is plotted against the travel
cost \(c_{ij}\) (x-axis), the probability of a trip occurring between a
given range of \(c_{ij}\) is the area under the curve. In this case, the
total area under the PDF curve always sums to 1, meaning that there is
100\% probability that the trip will occur between the minimum and
maximum \(c_{ij}\).

Dunn et al. \citeyearpar{dunn2023} presented a set of distributions that
serve as PDFs. From their survey, we selected some options for \(f()\)
commonly used in accessibility research and their impact on the number
of opportunities (the sum of opportunities) at specific travel costs
\(c_{ij}\), namely: uniform, negative exponential, gamma, normal, and
lognormal distributions.

\begin{itemize}
\tightlist
\item
  \textbf{Uniform distribution}: The uniform distribution or rectangular
  PDF looks very similar to the binary function, since it only returns
  one of two values, but ensure that area under the curve for the range
  of \(c_{ij}\) is 1. The uniform distribution PDF is shown in (Equation
  \ref{eq:uniform-equation}). The parameters to be calculated are
  \(c_{max}\) and \(c_{min}\), which represent the maximum and minimum
  travel costs that describe the observed or assumed willingness to
  reach destinations. In this distribution, all values within the
  interval are equally likely, and all values outside the interval have
  probability 0, assuming that the population's potential to interact
  with these opportunities is zero. Usually, \(c_{min}\) has value 0.
\end{itemize}

\begin{equation}
f(c_{ij})^{uniform} =
\begin{cases}
  \frac{1}{c_{max} - c_{min}} & \text{for } c_{min} \le c_{ij} \le c_{max} \\
  0 & \text{otherwise}
\end{cases}
\label{eq:uniform-equation}
\end{equation}

\begin{itemize}
\tightlist
\item
  \textbf{Exponential distribution}: The exponential distribution PDF
  equation is given by Equation \ref{eq:exponential-equation}. This
  model suggests that impedance decreases exponentially with increasing
  cost \((c_{ij})\). The parameter \(\beta\) represents the decay rate,
  with higher values indicating a faster decrease in accessibility with
  increasing cost. As already mentioned, this function is widely used
  due to its simplicity and ability to model the rapid drop-off in
  accessibility over distance.
\end{itemize}

\begin{equation}
f(c_{ij}) = e^{-\beta c_{ij}} \text{ with } c_{ij} \ge 0
\label{eq:exponential-equation}
\end{equation}

\begin{itemize}
\tightlist
\item
  \textbf{Gamma distribution}: The gamma distribution PDF equation is
  presented by the Equation \ref{eq:gamma-equation}. Where
  \(\Gamma(\alpha)\) is the gamma function to be estimated. In this
  case, the probability is typically low at low cost, higher at medium
  cost, and low again at high cost. The higher the \(\sigma\) (scale
  rate) parameter, the higher the probability that the majority of trips
  will be in the low cost range. So at low values of the \(\sigma\)
  (scale rate) parameter, the same probability is spread over a wider
  range of travel costs. For the \(\alpha\) (shape) parameter, the
  higher the value, the higher the probability density of trips with a
  higher average cost \citep{soukhov2024}.
\end{itemize}

\begin{equation}
f(c_{ij}) = 
   \begin{cases}
\frac{1}{\sigma^\alpha\Gamma(\alpha)} c_{ij}^{\alpha-1} e^{\frac{-c_{ij}}{\sigma}} & \text{if } 0 \leq c_{ij} <      \infty  \text{ and } \alpha, \sigma > 0 \\ 0 & \text{otherwise}
   \end{cases}
\label{eq:gamma-equation}
\end{equation}

\begin{itemize}
\tightlist
\item
  \textbf{Lognormal distribution}: The normal distribution, also often
  called the Gaussian distribution, is suitable when the travel cost is
  found to be distributed normally. The normal distribution has the PDF
  form displayed in Equation \ref{eq:normal-equation}. In this equation,
  \(\mu\) and \(\sigma\) are the mean and standard deviation of the
  distribution and need to be estimated together to control the shape of
  the normal curve. In this distribution, about 68\% of the observations
  will fall within 1 standard deviation of the mean, about 95\% will
  fall within 2 standard deviations, and about 99.7\% will fall within 3
  standard deviations of the mean. In this case, the values close to the
  mean will have the highest probability.
\end{itemize}

\begin{equation}
f(c_{ij}) = \frac{1}{\sqrt{2\pi} \sigma c_{ij}} e^{-\frac{(\ln c_{ij} - \mu)^2}{2\sigma^2}}
\label{eq:normal-equation}
\end{equation}

\begin{itemize}
\tightlist
\item
  \textbf{Lognormal distribution}: In many cases, the logarithm of the
  travel cost is found to be distributed normally. The lognormal
  distribution has the PDF form displayed in Equation
  \ref{eq:lognormal-equation}. It this equation, \(\mu\) and \(\sigma\)
  are the mean and standard deviation of the logarithm, and need to be
  estimated for together control the shape of the log-normal curve.
  Similar to the gamma function, the probability is typically low at low
  cost, higher at medium cost, and low again at high cost.
\end{itemize}

\begin{equation}
f(c_{ij}) = \frac{1}{\sqrt{2\pi} \sigma c_{ij}} e^{-\frac{(\ln c_{ij} - \mu)^2}{2\sigma^2}}
\label{eq:lognormal-equation}
\end{equation}

To identify our PDFs, we applied the \texttt{fitdistrplus} package
\citep{delignette2015fitdistrplus} to calculate the best PDF for every
destination, mode of transportation and survey year, modelling the trip
durations and testing the equations mentioned above. We selected the
best impedance function based on the lowest Akaike Information Criterion
(AIC) value \citep{akaike1974}. The AIC metric not only assesses the
goodness of fit but also penalizes model complexity to prevent
overfitting. AIC provides a balance between a model's accuracy and
simplicity, with lower values indicating a more economical model. The
distribution with the lowest AIC was considered the most suitable for
representing the distance decay curve for each specific destination in
each year. We chose AIC as the selection criterion because, while the
\texttt{fitdistrplus} package accommodates weighted episodes during
estimation, it does not extend this functionality to diagnostic plots,
which are typically unweighted and traditionally used to select the
best-fitting function.

In order to calculate the impedance functions, two filters were applied
in the TUS data set. The first is that we excluded all trips with travel
times higher than 100 minutes (1.5 hours). An exploratory data analysis
showed that, taking into account all the walking and cycling records
(31,761 in total), less than 0.58\% of them have a trip duration higher
than this limit. When considering the weights of this episodes, travel
times higher than 100 minutes represented 0.72\% of the episodes. Due to
the description of the activities, it was also possible to know that
trips with a duration higher than 100 minutes are mainly composed of
hiking and camping episodes. The second filter was realized to select
only the population living in a larger urban population centre. We
decided to apply this restriction because the travel behaviour of
residents of CMA and CA areas tends to be very different from those
outside these large urban centres in terms of active travel.

\section{Results and discussion}\label{results-and-discussion}

\subsection{Descriptive analysis}\label{descriptive-analysis}

\subsubsection{Active transportation
episodes}\label{active-transportation-episodes}

After applying the filters to the GSS surveys, we obtained a total of
21,456 cases of active travel episodes. However, GSS surveys apply a
probability sampling methodology, in which each episode or person
selected in the sample represents several other episodes or persons not
in the sample. The number of episodes and persons represented by a
episode or person is determined by the weight or weighting factor.
Because of this, every estimates of the number of episodes or persons
need to be calculated applying the corresponding weighting factors.

Considering the weights, the 21,456 episodes represent a total of
46,758,155 episodes. Table \ref{tab:episodes-count-percentages} contains
the weighted number of episodes about walking and cycling trips between
1992 and 2022. The year 2010 is the year with the most episodes, with
9,951,317 episodes (representing 21.28\% of the total). The year 1992
has the lowest number of episodes, with only 6,636,740 episodes,
representing 14.19\% of the total. The most recent survey, 2022, counted
with 6,205,495 episodes (13.27\% of the total).

When analyzing the two AT modes, walking episodes account for 93.46\%,
while the remaining 6.54\% are cycling episodes. The most recent survey
(2022) showed that cycling trips accounted for almost 8\% of AT
representation, the highest participation in all survey-years,
reinforcing a trend of increasing cycling participation that started in
1998.

\begin{table}
\centering
\caption{\label{tab:bulding table-01}\label{tab:episodes-count-percentages}Weighted number of episodes identified in each active transportation mode by year.}
\centering
\fontsize{10}{12}\selectfont
\begin{tabular}[t]{llrl>{}r|lr}
\toprule
\multicolumn{1}{c}{ } & \multicolumn{2}{c}{Cycling} & \multicolumn{2}{c}{Walking} & \multicolumn{2}{c}{Active Trips} \\
\cmidrule(l{3pt}r{3pt}){2-3} \cmidrule(l{3pt}r{3pt}){4-5} \cmidrule(l{3pt}r{3pt}){6-7}
Year &  & (\%) &  & (\%) &  & (\%)\\
\midrule
1992 & 468,438 & 7.06 & 6,168,301 & 92.94 & 6,636,740 & 14.19\\
1998 & 365,471 & 4.35 & 8,031,199 & 95.65 & 8,396,670 & 17.96\\
2005 & 564,468 & 6.89 & 7,625,937 & 93.11 & 8,190,405 & 17.52\\
2010 & 644,341 & 6.47 & 9,306,976 & 93.53 & 9,951,317 & 21.28\\
2015 & 534,187 & 7.24 & 6,843,342 & 92.76 & 7,377,529 & 15.78\\
\addlinespace
2022 & 481,639 & 7.76 & 5,723,857 & 92.24 & 6,205,495 & 13.27\\
Total & 3,058,544 & 6.54 & 43,699,611 & 93.46 & 46,758,155 & 100.00\\
\bottomrule
\end{tabular}
\end{table}

Figure \ref{fig:figure-destmodeyearperc} shows the percentage of each
destination by year and by mode of transport. Across all years analyzed,
`Home' is the most common travel destination, regardless of whether of
the active mode, with levels consistently above 33\%. `Work or school'
appears as the second most common destination in more recent years,
particularly for cycling trips, which peaked at almost 31\% in 2015. Up
to 2010, the TUS included fewer destination options, especially in 1992
and 1998. For this reason, the destination category `Elsewhere' - which
encompasses all destinations not explicitly listed by respondents - had
a very high representation for both transportation modes, ranking as the
second most common destination in the first two surveys (1992 and 1998).
However, as later TUS cycles introduced more detailed destination
categories, `Elsewhere' gradually lost its representativeness over time
and is no longer among the five most common destinations in 2022. Along
with the two destinations already mentioned, `Other's home' is the only
other destination present in the TUS surveys since 1992. This last
destination seems to be a destination with a higher share when it comes
to walking trips, but for both modes of transportation it seems that
respondents are going less and less to other people's homes - a fact
that can be explained by new communication technologies, in which a
person does not need to visit another person's home to keep in touch
with them.

\begin{figure}
\includegraphics[width=1\linewidth]{figures/destination_percentual} \caption{Percentage of walking and cycling trips categorized by destination and year.}\label{fig:figure-destmodeyearperc}
\end{figure}

After 2005, the expansion of the destination highlights some new popular
locations. For example, `Grocery store' appears as the third most chosen
destination, varying from around 9\% in 2005 to almost 15\% in 2022 for
cycling trips and from 10\% to 11.5\% for walking trips. When
considering walking trips, `Restaurants' appears as another well chosen
destination and, in the case of cycling trips, `Outdoors' appears as a
well chosen destination.

The maximum time spent on walking trips varied between 90 and 100
minutes across the years (Table \ref{tab:decrip-analysis}). It is
important to note that trips lasting more than 100 minutes were excluded
from the analysis. The mean walking time also varied, starting at 15
minutes in 1992, dropping to 11 minutes in 1998, and then increasing
steadily in subsequent surveys, reaching 18 minutes in 2022. However,
since the mean is highly influenced by extreme values, we also examine
the median travel time, which better represents a typical walking trip.
The median time spent walking remained constant at 10 minutes across the
early TUS cycles but increased to 15 minutes in 2022.

\begin{table}
\centering
\caption{\label{tab:table-02}\label{tab:decrip-analysis}Descriptive statistics for episodes with active transport records.}
\centering
\fontsize{10}{12}\selectfont
\begin{tabular}[t]{>{}llcccccc}
\toprule
\multicolumn{2}{c}{ } & \multicolumn{6}{c}{Year} \\
\cmidrule(l{3pt}r{3pt}){3-8}
Mode & Statistic & 1992 & 1998 & 2005 & 2010 & 2015 & 2022\\
\midrule
 & Maximum & 90 & 100 & 100 & 100 & 95 & 90\\

 & Mean & 15 & 11 & 12 & 12 & 16 & 18\\

 & Median & 10 & 10 & 10 & 10 & 10 & 15\\

 & Minimum & 3 & 1 & 1 & 1 & 5 & 5\\

\multirow[t]{-5}{*}{\raggedright\arraybackslash \textbf{Walking}} & Standard deviation & 14 & 11 & 11 & 13 & 13 & 15\\
\cmidrule{1-8}
 & Maximum & 100 & 90 & 95 & 100 & 90 & 60\\

 & Mean & 19 & 26 & 20 & 17 & 25 & 30\\

 & Median & 15 & 15 & 15 & 10 & 20 & 30\\

 & Minimum & 5 & 2 & 1 & 2 & 5 & 5\\

\multirow[t]{-5}{*}{\raggedright\arraybackslash \textbf{Cycling}} & Standard deviation & 20 & 19 & 15 & 16 & 16 & 14\\
\bottomrule
\end{tabular}
\end{table}

For cycling trips, the median travel time fluctuated across the survey
periods, remaining at 15 minutes from 1992 to 2005 and peaking at 30
minutes in 2022. As was also the case with the walking travel times,
these results show a trend of increasing for cycling trip duration
throughout the years. The analysis of travel time statistics alone does
not fully explain the reasons behind these fluctuations in travel time
over the years. However, it is likely that these variations reflect
changes in bicycle technology or cyclist behaviour.

Figure \ref{fig:figure-boxplot} presents box plots showing the
distribution of travel times for active transport modes over the years,
categorized by destination. The typical duration of walking trips was
consistently shorter than that of cycling trips. While we can compare
the temporal evolution of travel times, some destinations appear only in
the two most recent surveys, such as ``Neighborhood,'' ``Health
clinic,'' ``Sports area,'' and ``Business.'' The first three showed a
constant median walking travel time of 10 minutes in both surveys, while
the median travel time to ``Business'' increased from 10 to 15 minutes.
For cycling trips, ``Business'' recorded no trips, while
``Neighborhood'' had a typical travel time of 30 minutes in 2015 but no
records for 2022. ``Health clinic'' showed a constant cycling travel
time of 15 minutes, and ``Sports area'' doubled its typical duration,
from 15 to 30 minutes.

\begin{figure}
\includegraphics[width=1\linewidth]{figures/destination_boxplots} \caption{Percentage of walking trips categorized by origin and destination.}\label{fig:figure-boxplot}
\end{figure}

For the other destinations, starting with walking trips, we note a trend
of increasing travel times for almost all destinations, with an increase
observed at least in the most recent survey (2022). ``Restaurants'' and
``Outdoors'' both increased their typical travel time from 5 minutes in
2005 to 10 minutes in 2022; ``Other's home'' rose to 10 minutes in 2022
after remaining at 5 minutes since 1992; ``Place of worship'' increased
from 10 minutes in 2005 to 20 minutes in 2022; and ``Cultural venues''
rose from 10 minutes in 2005 to 20 minutes in 2022. The three most
popular types of destinations - ``Home,'' ``Work or school,'' and
``Grocery store'' - had an increase to 15 minutes after decades of
stabilization at 10 minutes. For both multimodal transportation options,
the median duration remained constant at 10 minutes, regardless of
whether the motorized mode involved private vehicles or public transit.

In general, while ``Place of worship'' and ``Cultural venues'' displayed
the highest median travel times of 20 minutes, the overall median
walking time cutoff across all surveys appears to be 10 minutes, with
most trips occurring below this threshold - except for 2022, when this
limit was increased by 15 minutes. No destination shows a decrease in
typical (median) travel time.

For cycling trips, only ``Cultural venues'' did not show an increase in
typical travel time when comparing 2022 to the previous years. In this
case, the travel time dropped from 25 minutes in 2010 to 15 minutes in
2015, although it remained higher than the 2005 value (10 minutes), and
no trips were recorded in the most recent survey (2022). ``Other's
home'' is the other destination with no cycling records for the 2022
survey. An increasing trend in travel times is evident for destinations
such as ``Grocery store'' (rising from a median of 10 to 60 minutes
between 2005 and 2022) and ``Restaurant'' (rising to a median of 30
minutes in 2022). However, unlike walking, cycling trips connecting to
motorized transportation increased from 5 minutes in 1992 to 10 minutes
in 2010, and then to 30 minutes in 2022. In contrast, cycling trips
connecting to public transit decreased from 20 minutes in 2015 to 12.5
minutes in 2022.

Other destinations seem to follow a similar pattern of increasing travel
time, where higher values were recorded in earlier survey cycles,
dropped over time, and then rose again in the most recent surveys. This
is the case for ``Home,'' which reached its highest typical travel time
of 25 minutes after dropping to 10 minutes in 2010. It is also worth
mentioning ``Work or school,'' which had a typical cycling travel time
of 15 minutes in 1992, peaked at 30 minutes in 1998, dropped back to 15
minutes in 2005 and 2010, and then increased to 25 minutes in 2022.

Figures \ref{fig:walking-heatmap} and \ref{fig:cycling-heatmap} show
walking and cycling trips from 2005 to 2022 using heat maps. These maps
use color gradients to represent the percentage of trips between origins
and destinations, with darker colors indicating higher percentages and
lighter colors representing less frequent routes. Since the two initial
surveys had only a few location options, trips that started or ended at
`Elsewhere' accounted for more than 50\% of cycling trips in 1992 and
1998, and over 70\% and 92\% of walking trips in those same years. This
representation gradually decreased to less than 7\% in 2022, as more
recent surveys included a wider range of origins and destinations.

After 2005, trips that started or ended at home represented more than
66\% of walking trips, with a particular highlight in 2022, when `Home'
served as a hub for almost 83\% of them. For cycling, `Home' was a hub
for over 90\% of trips since 2005, indicating that most cycling episodes
refer to people leaving or returning home. `Work or school' was the
second most important hub, especially for cycling trips, which ranged
from 62\% in 2015 to 48\% in 2022. For walking, this hub was the origin
or destination of almost 40\% of trips in 2022, showing an increase
since 2010, when it stood at 32\%.

We expected a higher representation of AT episodes starting or ending at
public transit hubs. For walking trips, this accounted for around 4\% in
2022 - slightly lower than for motorized vehicles, which reached 5.7\%.
For cycling, there is a trend of people using bicycles to connect to
transit, increasing from 0.2\% in 2005 to 1.2\% in 2022 - a small but
sixfold increase.

When comparing different combinations of origins and destinations, the
most common trips were those between `Home' and `Work or Study', which
increased from 12\% in 2010 to 25\% in 2022 for walking trips and
fluctuated between 33\% and 42\% for cycling trips. The second most
frequent type was between `Home' and grocery stores, which rose from
about 11\% in 2010 to 17\% in 2022 for both modes. Interestingly, trips
starting and ending at home have remained stable at around 6\% since
2005 for walking, but decreased steadily for cycling to around 1\% in
2022. This suggests that leisure trips, such as activities around the
home, are predominantly done on foot rather than by bicycle.

\begin{figure}
\includegraphics[width=1\linewidth]{figures/walking_hm_fig_v2} \caption{Percentage of walking trips categorized by origin and destination (only combinations exceeding 1\% of total trips are labeled).}\label{fig:walking-heatmap}
\end{figure}

\begin{figure}
\includegraphics[width=1\linewidth]{figures/cycling_hm_fig_v2} \caption{Percentage of walking trips categorized by origin and destination (only combinations exceeding 1\% of total trips are labeled).}\label{fig:cycling-heatmap}
\end{figure}

We analyzed whether the temporal differences in travel times for the
destinations were statistically significant. Only destinations that
appear in more than one survey year can have their temporal evolution
analyzed. Therefore, out of the twelve possible destinations, some
cycling locations could not be temporally analyzed: ``Business,''
``Neighborhood,'' and ``Place of worship.'' In the case of walking
trips, all destinations could be analyzed over time.

After performing the Kruskal-Wallis test (to assess whether there was a
statistically significant difference between the distributions of
empirical travel time values, considering the time differences for each
destination and the weight of each episode) and the pairwise Wilcoxon
test, we were able to identify the destinations where a statistically
significant difference was detected. For both AT modes, the possible
destinations had at least two year with statistically significant
difference in travel times (p-value \textless{} 0.05, Table A.1).
Considering the cycling mode and, for instance, the ``Home''
destination, there was a statistically significant difference for every
possible combination of two survey cycles. This result indicates that
the previously discussed increase in typical cycling travel time for
home destinations when compared 2022 to 1992 is statistically
significant.

\subsubsection{Population with records of active
trip}\label{population-with-records-of-active-trip}

The share of the population with active trip records decreased
significantly since 1992, when it reached almost 25\%, to less than
10.5\% in 2022 (Table \ref{tab:pop-stats-table}). This decline in the
representation of people with AT episodes is evident in both modes, but
it is more highlighted for walking trips, which fell from around 23\% in
1992 to about 9.5\% in 2022. In the last survey year, just
\emph{2,873,878} people out of a total population of \emph{27,584,823}
reported having at least one AT episode.

\begin{landscape}\begingroup\fontsize{10}{12}\selectfont

\begin{longtable}[t]{>{}lrrrrr>{}r|rrrrr>{}r|rrrrrr}
\caption{\label{tab:pop-table-with-prevalence}\label{tab:pop-stats-table}Prevalence of active trip by transportation mode, year of analysis, gender and age group.}\\
\toprule
\multicolumn{1}{c}{ } & \multicolumn{6}{c}{Walking} & \multicolumn{6}{c}{Cycling} & \multicolumn{6}{c}{Both modes} \\
\cmidrule(l{3pt}r{3pt}){2-7} \cmidrule(l{3pt}r{3pt}){8-13} \cmidrule(l{3pt}r{3pt}){14-19}
  & 1992 & 1998 & 2005 & 2010 & 2015 & 2022 & 1992 & 1998 & 2005 & 2010 & 2015 & 2022 & 1992 & 1998 & 2005 & 2010 & 2015 & 2022\\
\midrule
\textbf{Total} & 23.27 & 23.88 & 16.46 & 18.20 & 12.69 & 9.63 & 1.76 & 1.18 & 1.09 & 1.26 & 1.05 & 0.90 & 24.48 & 24.80 & 17.38 & 19.11 & 13.47 & 10.42\\
\textbf{Men+} & 22.46 & 22.00 & 14.93 & 17.66 & 12.51 & 9.51 & 2.58 & 1.95 & 1.71 & 2.02 & 1.43 & 1.33 & 24.19 & 23.55 & 16.40 & 19.11 & 13.63 & 10.68\\
\textbf{Women+} & 24.08 & 25.72 & 17.94 & 18.71 & 12.87 & 9.75 & 0.94 & 0.42 & 0.50 & 0.54 & 0.68 & 0.47 & 24.78 & 26.02 & 18.32 & 19.11 & 13.32 & 10.16\\
\midrule
\textbf{15 - 24} & 33.78 & 36.67 & 28.79 & 27.89 & 19.19 & 21.77 & 5.58 & 2.23 & 2.28 & 2.79 & 1.07 & 1.58 & 37.74 & 38.26 & 30.72 & 29.66 & 19.77 & 23.36\\
\textbf{25 - 34} & 21.91 & 23.38 & 16.85 & 20.09 & 17.43 & 10.68 & 2.14 & 0.86 & 1.23 & 1.32 & 1.60 & 1.62 & 23.46 & 23.97 & 17.74 & 21.08 & 18.44 & 12.15\\
\midrule
\addlinespace
\textbf{35 - 44} & 19.47 & 19.73 & 13.88 & 17.28 & 11.14 & 6.31 & 0.75 & 2.01 & 1.12 & 1.18 & 1.22 & 0.83 & 19.85 & 21.40 & 14.84 & 18.16 & 12.26 & 6.86\\
\textbf{45 - 54} & 20.79 & 20.11 & 12.31 & 14.82 & 9.89 & 7.18 & 0.30 & 0.82 & 0.61 & 1.00 & 0.89 & 0.84 & 20.98 & 20.80 & 12.84 & 15.65 & 10.50 & 7.80\\
\textbf{55 - 64} & 21.34 & 19.49 & 12.89 & 14.72 & 9.34 & 6.29 & 0.46 & 0.44 & 0.95 & 0.93 & 0.82 & 0.47 & 21.59 & 19.93 & 13.76 & 15.51 & 10.08 & 6.75\\
\textbf{65 - 74} & 21.64 & 21.63 & 11.67 & 14.31 & 9.41 & 7.01 & 0.00 & 0.00 & 0.36 & 0.42 & 0.82 & 0.33 & 21.64 & 21.63 & 12.03 & 14.53 & 10.06 & 7.31\\
\textbf{75+} & 21.87 & 26.23 & 16.10 & 13.69 & 9.41 & 7.00 & 0.00 & 0.00 & 0.12 & 0.11 & 0.57 & 0.09 & 21.87 & 26.23 & 16.22 & 13.80 & 9.98 & 7.08\\
\bottomrule
\end{longtable}
\endgroup{}
\end{landscape}

Our results show a decline in the active population since 1992 for both
genders and for both modes of transportation (Table
\ref{tab:pop-stats-table} and Figure \ref{fig:gender-perc-figure}).
However, the decrease was more pronounced among women+ than men+,
leading to a shift in the historical pattern in which women+ had
traditionally been the more active gender in Canada
\citep{bryan2009, borhani2024}. In 1992, 24.19\% of women+ had at least
one AT episode, but this dropped to a low of 10.16\% in 2022. Among
men+, the share fell from 24.19\% in 1992 to 10.68\% in 2022, making
men+ the gender with the higher share of the active population in the
most recent survey year. This result aligns with a trend identified by
Borhani et al. \citeyearpar{borhani2024}. In their study, the authors
found that a higher proportion of females than males reported engaging
in any form of AT, whether walking or cycling, in the last 7 days or 3
months. However, they also observed that this gender gap appears to have
narrowed over time.

When analyzing by transportation mode, in all survey years a higher
proportion of men+ reported at least one cycling episode on the previous
day compared to women+ (ranging from 1.33\% to 2.58\% for men+ and
0.47\% to 0.94\% for women+). This pattern of greater men+ participation
in cycling is consistent with previous research
\citep{heesch2012, bryan2009, borhani2024}. Conversely, women+ have
historically reported more walking trips than men+, but this pattern
appears to be changing, also in line with prior findings
\citep{goel2023, pollard2017, bryan2009, borhani2024}, suggesting the
possibility of a reversal in the next survey cycle.

\begin{figure}
\includegraphics[width=1\linewidth]{figures/active_pop_gender_graph} \caption{Prevelance in activity transportation by gender.}\label{fig:gender-perc-figure}
\end{figure}

The decreasing of AT participation when compared to ealier surveys is
perceptible for all group ages (Figure \ref{fig:age-perc-figure}). The
youngest group (those between 15 and 24 years) stand out as the most
active group, ranging from 20\% to 37\%, and marking 23\% in the most
recent survey. This is the only group that did not show a decrease in
the prevalence of active participation in 2022 when compared to 2015 -
although still below the levels recorded in 2010 (30\%). The survey of
2022 is one more evidence of caracterists of the Generation z (born
between 1997 and 2012) \citep{dimock2019} are noticeably less dependent
on cars and instead use environmentally friendly modes of travel, such
as public transport, cycling and walking, more often than not
\citep{haseeb2024, grimsrud2014, kuhnimhof2011}. Historically, and in
consistency with the literature \citep{bryan2009, borhani2024},
prevalence decreases as age increases. However, in the most recent
survey (2022), the oldest group (75 years and older) presented the
fourth-highest prevalence (7.08\%), surpassing younger groups.

The analysis by mode shows a similar trend (Figure
\ref{fig:age-perc-figure}). However, for cycling, the second youngest
group (aged 25 to 34 years) had the highest prevalence in the 2022
survey (1.60\%), surpassing the youngest group (1.58\%). For all other
age groups, cycling prevalence decreases as age increases, approaching
0.10\% for the oldest group (75 years and older).

\begin{figure}
\includegraphics[width=1\linewidth]{figures/active_pop_age_graph} \caption{Prevelance in activity transportation by age group.}\label{fig:age-perc-figure}
\end{figure}

The number of people with active trip episodes is primarily influenced
by walking, since over 92\% of all recorded active trips involve
individuals with walking episodes. When analyzing the average number of
episodes per person with AT records, the mean is approximately 2.3
active episodes per person, ranging from a maximum of 2.37 episodes in
1992 to a minimum of 2.19 episodes in 2022. Figure
\ref{fig:gender-eps-figure} presents, for each survey year, the total
number of active episodes divided by the population with active records,
as well as the number of walking episodes per person who walked and the
number of cycling episodes per person who cycled, all disaggregated by
gender. Historically, men+ recorded more active episodes than women+,
but this pattern appears to be shifting. While men+ decreased from 2.51
episodes in 1992 to 2.22 in 2022, women+ also experienced a decline,
though smaller, from 2.24 to 2.17 episodes. This shift is mainly caused
by a reduction in walking episodes among men+, combined with an increase
in both walking and cycling episodes among women+ in 2022.

\begin{figure}
\includegraphics[width=1\linewidth]{figures/eps_gender_graph} \caption{Episodes per active person by gender.}\label{fig:gender-eps-figure}
\end{figure}

Similar to gender patterns, all age groups performed worse in 2022
compared to 1992. However, Figure \ref{fig:age-eps-figure} shows that
only the two youngest groups (those under 34 years old) did not increase
their number of episodes per person in 2022 when compared to 2015. Our
results indicate that the most active age group in 2022 was those aged
55 to 64 years, with 2.51 episodes per active person. Active individuals
in the oldest age group (75 and over) have shown a consistent upward
trend in the number of active episodes since 2005. In that year, this
group ranked last in the number of active episodes per active person
(2.08), but by 2022, it had risen to second place, with 2.26 episodes
per active person (a 9\% increase). When the age group analysis is
disaggregated by transportation mode, walking episodes per walker follow
a pattern similar to that of total active episodes per active person
(Figure \ref{fig:age-eps-figure}). However, for cycling, the youngest
group has shown an increasing trend in cycling episodes per cyclist,
rising from 1.98 in 1998 to 3.00 in 2022 --- a 51\% increase.

\begin{figure}
\includegraphics[width=1\linewidth]{figures/eps_age_graph} \caption{Episodes per active person by age group.}\label{fig:age-eps-figure}
\end{figure}

Regarding the duration of AT episodes, we observe an overall increase in
the typical duration of episodes for both genders and both
transportation modes across the years (Figure
\ref{fig:gender-dur-figure}). The median duration of walking trips
remained stable at 10 minutes for both men+ and women+ from 1992 to
2015, increasing to 15 minutes in 2022. For cycling trips, however, men+
reported a median cycling duration of 30 minutes in 2022, up from 15
minutes in 2005, marking the highest median duration recorded in the
entire historical series. In contrast, women+ saw a decrease in their
median duration, from 20 minutes in 2015 to 15 minutes in 2022.

\begin{figure}
\includegraphics[width=1\linewidth]{figures/gender_duration_boxplots_sample} \caption{Duration (in minutes) of active episodes by gender.}\label{fig:gender-dur-figure}
\end{figure}

An examination by age group reveals a general increase in AT trip
duration across all groups since 1998 (Figure \ref{fig:age-dur-figure}).
With the exception of the 65 to 74 age group, for which the typical
walking duration remained constant at 10 minutes, all other groups
increased their typical walking duration to 15 minutes. Regarding
cycling trips, two distinct periods can be identified: from 1992 to
2010, when the median duration decreased, and from 2010 to 2022, when
the typical duration increased across all age groups.

\begin{figure}
\includegraphics[width=1\linewidth]{figures/age_duration_boxplots} \caption{Duration (in minutes) of active episodes by Age group.}\label{fig:age-dur-figure}
\end{figure}

\subsection{Calibrated impedance
function}\label{calibrated-impedance-function}

In total, we fitted 114 impedance functions considering different
destinations, AT mode, and survey-years. Among the candidate
distributions, only the negative exponential type was not fitted as the
best option. The absence of exponential functions, given the variety of
destinations, year and mode of transport, indicates that the impedance
functions applied in active accessibility studies may not be adequately
measuring travel behaviour, especially for cases when the travel time is
close to 0 minute. Table A.2 displays the selected functions for walking
trips, while Table A.3 presents the functions for cycling trips.

Figure \ref{fig:outdoors-impedance-fig} shows the calibrated functions
for the destination `Outdoors,' along with a histogram of the empirical
distribution of trips, split by year and transportation mode. Starting
with the functions from the walking transportation mode (blue curves),
the calibrated functions from this example show a similar pattern. At a
duration of around zero minutes, the probability of making the trip is
lower (with a density of zero for the years 2015 and 2022). After a few
minutes, there is a peak in the maximum probability of traveling to
reach `Outdoors,' followed by a drop in willingness to zero for very
high values of time, indicating a low probability of making the trip.

\begin{figure}

{\centering \includegraphics[width=1\linewidth]{figures/impf_Outdoors} 

}

\caption{Empirical data and impedance functions fitted for walking trips for Outdoors destination.}\label{fig:outdoors-impedance-fig}
\end{figure}

For 2010, the selected impedance function has a gamma form, with a shape
parameter of \(\alpha = 1.07\) and a rate parameter of
\(\sigma = 0.11\). The rate parameter primarily controls the steepness
of the curve's decline, while the shape parameter determines how the
density peak shifts along the \(x\)-axis (travel time). A larger shape
value indicates that the probability peak occurs at higher travel times.
In 2010, the peak occurs at 1 minute.

For 2005, 2015, and 2022, the probability density functions (PDFs) that
best represent the population's transport behaviour are lognormal
distributions. In 2005, the distribution has a mean of \(\mu = 1.99\)
and a standard deviation of \(\sigma = 0.76\). In 2015, the mean
increases to \(\mu = 2.58\) with a standard deviation of
\(\sigma = 0.75\), and in 2022, the mean is \(\mu = 2.28\) with a
standard deviation of \(\sigma = 0.78\). In 2005, the density peak
occurs at a journey duration of 4 minutes (0.10), while in 2015, the
peak occurs at 8 minutes (0.05). A lower density peak corresponds to a
more dispersed curve, with higher densities at longer durations. While
walking trips in 2005 and 2022 have densities close to zero for
durations beyond 50 minutes, the 2015 curve shows a small density
(0.002) at the 50-minute mark. In 2022, the density peak occurs at 5
minutes (0.07), and the curve is less dispersed than in 2015.

For cycling trips, the best-fitting impedance function in 2005 is of the
gamma type, while in 2010 and 2022 lognormal distributions provide the
best fit. In 2015, the best-fitting PDF is a uniform distribution with
an upper bound of 35 minutes and a peak density of 0.028. The presence
of uniform distributions indicates that it was not possible to
parameterize more complex functions due to the low number of episodes
for this specific combination of destination, mode, and year, and in
this case, only three episodes were identified. Overall, all uniform
functions involve a maximum of 11 episodes, most of which correspond to
cycling, which accounts for only 7\% of all active travel episodes.
Cycling trips generally show greater dispersion and longer typical
durations compared to walking trips.

The complexity of the impedance function depends on the number of
episodes available for calibration. Fitting a gamma-type function
required an average of 193 episodes, while fitting a lognormal function
required approximately 227 episodes. In contrast, fitting a normal
function required only six episodes, and fitting a uniform function
required, on average, just 4 episodes.

The temporal evolution of the decay functions is illustrated in Figure
\ref{fig:walking-evolution-fig}, which shows calibrated functions for
each year across all destination and transport mode categories for
walking trips. For some destinations, the impedance functions are
consistent in type and parameters across all years analyzed. For
instance, trips to cultural venues consistently follow a gamma
distribution, with average trip durations increasing over time as peaks
shift to the right, a trend primarily captured by the rate parameter
(\(\sigma\)). In contrast, trips to places of worship show more
pronounced temporal variation, with noticeable changes in peak positions
and density dispersion. The emergence of uniform distributions for this
destination indicates that the total number of trips has declined over
time.

\begin{figure}

{\centering \includegraphics[width=1\linewidth]{figures/walking_temporal_evolution} 

}

\caption{Temporal evolution of walking impedance functions.}\label{fig:walking-evolution-fig}
\end{figure}

\section{Summary and conclusion}\label{summary-and-conclusion}

The main objectives of this study were to provide an overview of AT in
Canadian metropolitan cities, focusing on main origins, destinations,
travel times and active population on terms of age group and sex, and to
identify appropriate impedance functions for AT modes across various
destinations and time periods. In this study we perform a direct
application of \texttt{ActiveCA} R package \citep{dossantos2025},
analyzing over 13,500 cases of active travel trips that represented
46,758,155 episodes, from the Time Use cycles of the General Social
Survey (GSS) from 1992 to 2022, covering a twelve different type of
destinations and considering walking and cycling as transportation
modes.

The rate of active trips per person with active record was around two
trips, with an increasing trend in active episodes per person being
observed for both walking and cycling. Historically, men+ recorded more
active episodes than women+, but in 2022 this trend reversed: men+
averaged 2.07 episodes, while women+ averaged 2.11. This change was
driven by a decrease in walking episodes among men+ (from 2.16 in 2015
to 2.07 in 2022) and an increase among women+ (from 1.92 to 2.11 in the
same period).

Our results show that the typical duration of walking trips increased to
15 minutes by 2022, following years of stability at 10 minutes. For
cycling, the typical duration rose to 30 minutes, recovering from a
decline that began in 1998 and lasted until 2010. Generally, walking
trips had consistently shorter durations than cycling trips. Although
this study does not explain the causes of these fluctuations, the
differences by year and mode were statistically significant.

When analyzed by gender, men+ increased their median cycling trip
duration from 15 minutes in 2005 to 30 minutes in 2022. Women+ increased
their median from 10 to 15 minutes. No gender differences were
identified in the duration of walking trips. When analyzing by age
groups, all of them increased their active travel times compared to
1992, especially for walking. For cycling, a marked increase in duration
emerged after 2010.

For both transportation modes trips to ``Other's home'' declined over
time, likely reflecting changing social behaviour enabled by technology
that allows people to stay connected without visiting in person. Walking
trips were predominantly associated with ``Home'' as either the origin
or destination. For cycling, the combination of ``Home'' and ``Work or
school'' accounted for the majority of trips. Walking trip durations to
``Restaurants'' and ``Outdoors'' increased from 5 minutes in 2005 to 10
minutes in 2022. Travel to ``Places of worship'' also rose from 10
minutes in 2005 to 20 minutes in 2022, tying with ``Cultural venues''
for the highest typical walking travel time. Walking times to ``Home''
and ``Work or school'' increased to 15 minutes in 2022, following three
GSS cycles where they remained stable at 10 minutes. For cycling, 2015
marked the beginning of a trend toward longer travel times across nearly
all destinations.

The share of the population with at least one recorded active trip
ranged from a low of 6.93\% in 1998 to a peak of 15.06\% in 2010. In the
most recent GSS survey (2022), participation dropped to 9.45\%. Walking
dominated active trips, representing over 90\% of all episodes.

Since 2010, the active population has declined for both genders, with a
steeper decrease among women+. This reversed a long-standing pattern in
which women+ had higher AT participation than men+. In 2022, men+ showed
higher prevalence than women+ regardless of mode, with 10.05\% for men+
and 8.84\% for women+. In summary, women+ made more active trips, while
men+ had higher overall participation in AT.

Regarding age cohorts, the youngest group (15 to 24 years) remained the
most active, with a prevalence of 21.47\% in 2022. This was the only
group to show increased participation over the past decade, up from
17.91\% in 2015. Generally, AT prevalence decreases with age. However,
in both 2005 and 2022, the oldest group (75 years and older) reported
the third-highest AT prevalence (7.02\%) and showed a steady increase in
active episodes since 2010.

The study underscores the importance of applying destination-specific
impedance functions when measuring cost decay effects in accessibility
analyses. To this end, we fitted 83 impedance functions for AT trips
over a 30-year period, considering destination types and transportation
modes. The results indicate that none of the fitted functions followed
an exponential distribution, suggesting that commonly used functions in
AT accessibility studies may not adequately capture actual behaviour -
especially for very short trips (under 5 minutes), which tend to be
overrepresented in these models. Destinations with many episodes were
best modeled using gamma functions, followed by lognormal and normal
distributions. In contrast, destinations with fewer than six episodes
were best represented by uniform distributions.

Given similarities in urbanization processes between Canada, the United
States, Australia, and West Europe, these findings may also be
applicable to metropolitan areas in those regions. Finally, this study
contributes to the ongoing discussion on AT, emphasizing its importance
in promoting sustainable transportation planning.

\section*{CRediT authorship contribution
statement}\label{credit-authorship-contribution-statement}
\addcontentsline{toc}{section}{CRediT authorship contribution statement}

\emph{(include after the review)}

\section*{Funding sources}\label{funding-sources}
\addcontentsline{toc}{section}{Funding sources}

This research was funded through project \emph{project name provided
after review)}, supported by the Social Sciences and Humanities Research
Council of Canada.

\section*{Data availability}\label{data-availability}
\addcontentsline{toc}{section}{Data availability}

We updated the \texttt{ActiveCA} R Package to include the methodology to
obtain impedance functions from the raw data files (GSS surveys).
Additionally, we created this paper using literate programming in which
the R markdown code to fully reproduce this article is available on our
GitHub repository \emph{(include after the review)}.

\section*{Declaration of competing
interest}\label{declaration-of-competing-interest}
\addcontentsline{toc}{section}{Declaration of competing interest}

The authors declare no conflicts of interest.

\section*{Appendix}\label{appendix}
\addcontentsline{toc}{section}{Appendix}

\section*{References}\label{references}
\addcontentsline{toc}{section}{References}

\renewcommand{\thetable}{A.1}

\begingroup\fontsize{8}{10}\selectfont

\begin{longtable}[t]{llllllll}
\caption{\label{tab:kruskal-walking}\label{tab:result-stats}P-values of the pairwise Wilcoxon test.}\\
\toprule
Mode & Destination & Year & 1992 & 1998 & 2005 & 2010 & 2015\\
\midrule
 & Home & 1998 & 0.00e+00 &  &  &  & \\
\nopagebreak
 & Home & 2005 & 0.00e+00 & 1.00e+00 &  &  & \\
\nopagebreak
 & Home & 2010 & 0.00e+00 & 1.56e-290 & 0.00e+00 &  & \\
\nopagebreak
 & Home & 2015 & 0.00e+00 & 0.00e+00 & 0.00e+00 & 0.00e+00 & \\
\nopagebreak
 & Home & 2022 & 0.00e+00 & 0.00e+00 & 0.00e+00 & 0.00e+00 & 0.00e+00\\
\nopagebreak
 & Elsewhere & 1998 & 0.00e+00 &  &  &  & \\
\nopagebreak
 & Elsewhere & 2005 & 0.00e+00 & 2.56e-13 &  &  & \\
\nopagebreak
 & Elsewhere & 2010 & 0.00e+00 & 0.00e+00 & 7.70e-112 &  & \\
\nopagebreak
 & Elsewhere & 2015 & 0.00e+00 & 0.00e+00 & 0.00e+00 & 0.00e+00 & \\
\nopagebreak
 & Elsewhere & 2022 & 0.00e+00 & 0.00e+00 & 0.00e+00 & 0.00e+00 & 0.00e+00\\
\nopagebreak
 & Work or school & 1998 & 0.00e+00 &  &  &  & \\
\nopagebreak
 & Work or school & 2005 & 0.00e+00 & 0.00e+00 &  &  & \\
\nopagebreak
 & Work or school & 2010 & 6.02e-174 & 0.00e+00 & 0.00e+00 &  & \\
\nopagebreak
 & Work or school & 2015 & 0.00e+00 & 0.00e+00 & 0.00e+00 & 0.00e+00 & \\
\nopagebreak
 & Work or school & 2022 & 0.00e+00 & 0.00e+00 & 0.00e+00 & 0.00e+00 & 0.00e+00\\
\nopagebreak
 & Grocery store & 2010 &  &  & 0.00e+00 &  & \\
\nopagebreak
 & Grocery store & 2015 &  &  & 0.00e+00 & 0.00e+00 & \\
\nopagebreak
 & Grocery store & 2022 &  &  & 0.00e+00 & 0.00e+00 & 0.00e+00\\
\nopagebreak
 & Neighbourhood & 2022 &  &  &  &  & 0.00e+00\\
\nopagebreak
 & Sport area & 2022 &  &  &  &  & 0.00e+00\\
\nopagebreak
 & Outdoors & 2010 &  &  & 5.66e-09 &  & \\
\nopagebreak
 & Outdoors & 2015 &  &  & 0.00e+00 & 0.00e+00 & \\
\nopagebreak
 & Outdoors & 2022 &  &  & 0.00e+00 & 0.00e+00 & 0.00e+00\\
\nopagebreak
 & Travel - Transit & 1998 & 0.00e+00 &  &  &  & \\
\nopagebreak
 & Travel - Transit & 2005 & 0.00e+00 & 0.00e+00 &  &  & \\
\nopagebreak
 & Travel - Transit & 2010 & 7.45e-74 & 0.00e+00 & 0.00e+00 &  & \\
\nopagebreak
 & Travel - Transit & 2015 & 0.00e+00 & 0.00e+00 & 3.37e-46 & 0.00e+00 & \\
\nopagebreak
 & Travel - Transit & 2022 & 0.00e+00 & 0.00e+00 & 0.00e+00 & 0.00e+00 & 0.00e+00\\
\nopagebreak
 & Restaurant & 2010 &  &  & 0.00e+00 &  & \\
\nopagebreak
 & Restaurant & 2015 &  &  & 0.00e+00 & 0.00e+00 & \\
\nopagebreak
 & Restaurant & 2022 &  &  & 0.00e+00 & 0.00e+00 & 2.28e-164\\
\nopagebreak
 & Other's home & 1998 & 0.00e+00 &  &  &  & \\
\nopagebreak
 & Other's home & 2005 & 0.00e+00 & 0.00e+00 &  &  & \\
\nopagebreak
 & Other's home & 2010 & 0.00e+00 & 0.00e+00 & 2.43e-257 &  & \\
\nopagebreak
 & Other's home & 2015 & 0.00e+00 & 0.00e+00 & 0.00e+00 & 0.00e+00 & \\
\nopagebreak
 & Other's home & 2022 & 2.17e-100 & 0.00e+00 & 0.00e+00 & 0.00e+00 & 0.00e+00\\
\nopagebreak
 & Travel - Motorized & 1998 & 0.00e+00 &  &  &  & \\
\nopagebreak
 & Travel - Motorized & 2005 & 0.00e+00 & 0.00e+00 &  &  & \\
\nopagebreak
 & Travel - Motorized & 2010 & 0.00e+00 & 0.00e+00 & 0.00e+00 &  & \\
\nopagebreak
 & Travel - Motorized & 2015 & 9.95e-293 & 0.00e+00 & 0.00e+00 & 0.00e+00 & \\
\nopagebreak
 & Travel - Motorized & 2022 & 1.47e-26 & 0.00e+00 & 0.00e+00 & 0.00e+00 & 4.20e-56\\
\nopagebreak
 & Health clinic & 2022 &  &  &  &  & 1.12e-262\\
\nopagebreak
 & Cultural venues & 2010 &  &  & 1.63e-191 &  & \\
\nopagebreak
 & Cultural venues & 2015 &  &  & 1.63e-236 & 0.00e+00 & \\
\nopagebreak
 & Cultural venues & 2022 &  &  & 0.00e+00 & 0.00e+00 & 0.00e+00\\
\nopagebreak
 & Place of worship & 2010 &  &  & 0.00e+00 &  & \\
\nopagebreak
 & Place of worship & 2015 &  &  & 0.00e+00 & 0.00e+00 & \\
\nopagebreak
 & Place of worship & 2022 &  &  & 0.00e+00 & 0.00e+00 & 3.48e-01\\
\nopagebreak
\multirow[t]{-49}{*}{\raggedright\arraybackslash Walking} & Business & 2022 &  &  &  &  & 0.00e+00\\
\cmidrule{1-8}\pagebreak[0]
 & Travel - Motorized & 1998 & 2.76e-10 &  &  &  & \\
\nopagebreak
 & Travel - Motorized & 2005 & 0.00e+00 & 0.00e+00 &  &  & \\
\nopagebreak
 & Travel - Motorized & 2010 & 1.93e-90 & 1.91e-08 & 0.00e+00 &  & \\
\nopagebreak
 & Travel - Motorized & 2015 & 0.00e+00 & 0.00e+00 & 7.45e-220 & 0.00e+00 & \\
\nopagebreak
 & Travel - Motorized & 2022 & 0.00e+00 & 1.44e-164 & 5.61e-94 & 1.44e-119 & 2.61e-79\\
\nopagebreak
 & Grocery store & 2010 &  &  & 0.00e+00 &  & \\
\nopagebreak
 & Grocery store & 2015 &  &  & 0.00e+00 & 1.84e-242 & \\
\nopagebreak
 & Grocery store & 2022 &  &  & 0.00e+00 & 0.00e+00 & 0.00e+00\\
\nopagebreak
 & Home & 1998 & 0.00e+00 &  &  &  & \\
\nopagebreak
 & Home & 2005 & 2.62e-79 & 0.00e+00 &  &  & \\
\nopagebreak
 & Home & 2010 & 0.00e+00 & 0.00e+00 & 0.00e+00 &  & \\
\nopagebreak
 & Home & 2015 & 0.00e+00 & 0.00e+00 & 0.00e+00 & 0.00e+00 & \\
\nopagebreak
 & Home & 2022 & 0.00e+00 & 0.00e+00 & 0.00e+00 & 0.00e+00 & 0.00e+00\\
\nopagebreak
 & Work or school & 1998 & 0.00e+00 &  &  &  & \\
\nopagebreak
 & Work or school & 2005 & 1.35e-150 & 0.00e+00 &  &  & \\
\nopagebreak
 & Work or school & 2010 & 7.08e-167 & 0.00e+00 & 0.00e+00 &  & \\
\nopagebreak
 & Work or school & 2015 & 0.00e+00 & 0.00e+00 & 0.00e+00 & 0.00e+00 & \\
\nopagebreak
 & Work or school & 2022 & 0.00e+00 & 0.00e+00 & 0.00e+00 & 0.00e+00 & 0.00e+00\\
\nopagebreak
 & Health clinic & 2022 &  &  &  &  & 0.00e+00\\
\nopagebreak
 & Restaurant & 2010 &  &  & 1.60e-65 &  & \\
\nopagebreak
 & Restaurant & 2015 &  &  & 0.00e+00 & 0.00e+00 & \\
\nopagebreak
 & Restaurant & 2022 &  &  & 0.00e+00 & 0.00e+00 & 0.00e+00\\
\nopagebreak
 & Sport area & 2022 &  &  &  &  & 0.00e+00\\
\nopagebreak
 & Travel - Transit & 2022 &  &  &  &  & 0.00e+00\\
\nopagebreak
 & Outdoors & 2010 &  &  & 0.00e+00 &  & \\
\nopagebreak
 & Outdoors & 2015 &  &  & 0.00e+00 & 0.00e+00 & \\
\nopagebreak
 & Outdoors & 2022 &  &  & 0.00e+00 & 0.00e+00 & 0.00e+00\\
\nopagebreak
 & Elsewhere & 1998 & 0.00e+00 &  &  &  & \\
\nopagebreak
 & Elsewhere & 2005 & 0.00e+00 & 0.00e+00 &  &  & \\
\nopagebreak
 & Elsewhere & 2010 & 1.57e-84 & 0.00e+00 & 0.00e+00 &  & \\
\nopagebreak
 & Elsewhere & 2015 & 0.00e+00 & 0.00e+00 & 0.00e+00 & 0.00e+00 & \\
\nopagebreak
 & Other's home & 1998 & 1.58e-24 &  &  &  & \\
\nopagebreak
 & Other's home & 2005 & 0.00e+00 & 0.00e+00 &  &  & \\
\nopagebreak
 & Other's home & 2010 & 1.19e-16 & 0.00e+00 & 9.37e-196 &  & \\
\nopagebreak
 & Other's home & 2015 & 2.59e-42 & 1.33e-82 & 6.52e-171 & 1.84e-148 & \\
\nopagebreak
 & Cultural venues & 2010 &  &  & 2.98e-142 &  & \\
\nopagebreak
\multirow[t]{-37}{*}{\raggedright\arraybackslash Cycling} & Cultural venues & 2015 &  &  & 0.00e+00 & 9.65e-01 & \\
\bottomrule
\end{longtable}
\endgroup{}

\renewcommand{\thetable}{A.2}

\begin{table}
\centering
\caption{\label{tab:selected-functions}\label{tab:walking-functions-tab}Impedance functions and AIC for walking trips.}
\centering
\resizebox{\ifdim\width>\linewidth\linewidth\else\width\fi}{!}{
\fontsize{8}{10}\selectfont
\begin{threeparttable}
\begin{tabular}[t]{rllrrll}
\toprule
\multicolumn{1}{c}{\textbf{Year}} & \multicolumn{1}{c}{\textbf{Destination}} & \multicolumn{1}{c}{\textbf{Impedance function}} & \multicolumn{1}{c}{\textbf{Parameter 1*}} & \multicolumn{1}{c}{\textbf{Parameter 2*}} & \multicolumn{1}{c}{\textbf{AIC}} & \multicolumn{1}{c}{\textbf{Count}}\\
\midrule
 & Elsewhere & Lognormal & 2.29 & 0.70 & 14,353,151 & 763\\

 & Home & Lognormal & 2.66 & 0.79 & 17,710,035 & 781\\

 & Other's home & Lognormal & 2.24 & 0.81 & 2,625,543 & 127\\

 & Travel - Motorized & Lognormal & 2.46 & 0.72 & 1,622,219 & 70\\

 & Travel - Transit & Lognormal & 1.95 & 0.81 & 2,291,568 & 103\\

\multirow[t]{-6}{*}{\raggedleft\arraybackslash 1992} & Work or school & Lognormal & 2.19 & 0.71 & 4,329,953 & 225\\
\cmidrule{1-7}
 & Elsewhere & Lognormal & 2.06 & 0.81 & 24,191,134 & 1,307\\

 & Home & Lognormal & 2.28 & 0.81 & 17,603,632 & 915\\

 & Other's home & Lognormal & 1.87 & 0.94 & 3,945,176 & 230\\

 & Travel - Motorized & Lognormal & 1.89 & 0.76 & 1,293,887 & 68\\

 & Travel - Transit & Gamma & 1.53 & 0.21 & 2,065,748 & 107\\

\multirow[t]{-6}{*}{\raggedleft\arraybackslash 1998} & Work or school & Lognormal & 2.00 & 0.83 & 4,327,695 & 227\\
\cmidrule{1-7}
 & Cultural venues & Gamma & 3.38 & 0.30 & 258,623 & 26\\

 & Elsewhere & Lognormal & 2.09 & 0.83 & 4,696,559 & 535\\

 & Grocery store & Lognormal & 2.20 & 0.78 & 5,388,073 & 625\\

 & Home & Gamma & 1.19 & 0.09 & 20,075,099 & 2,133\\

 & Other's home & Lognormal & 1.89 & 0.83 & 3,780,185 & 489\\

 & Outdoors & Lognormal & 1.99 & 0.76 & 1,431,649 & 172\\

 & Place of worship & Gamma & 2.32 & 0.21 & 249,606 & 35\\

 & Restaurant & Lognormal & 1.89 & 0.81 & 4,693,138 & 533\\

 & Travel - Motorized & Lognormal & 2.31 & 0.77 & 1,172,262 & 119\\

 & Travel - Transit & Lognormal & 2.30 & 0.63 & 1,162,252 & 114\\

\multirow[t]{-11}{*}{\raggedleft\arraybackslash 2005} & Work or school & Lognormal & 2.13 & 0.78 & 9,163,421 & 819\\
\cmidrule{1-7}
 & Cultural venues & Gamma & 3.10 & 0.30 & 327,022 & 27\\

 & Elsewhere & Lognormal & 2.16 & 0.86 & 5,358,329 & 454\\

 & Grocery store & Lognormal & 2.00 & 0.93 & 7,585,227 & 581\\

 & Home & Gamma & 1.07 & 0.07 & 22,725,772 & 1,687\\

 & Other's home & Lognormal & 1.84 & 0.92 & 4,579,586 & 376\\

 & Outdoors & Gamma & 1.07 & 0.11 & 2,462,453 & 194\\

 & Place of worship & Lognormal & 1.96 & 0.69 & 295,149 & 29\\

 & Restaurant & Lognormal & 1.98 & 0.89 & 6,340,595 & 468\\

 & Travel - Motorized & Lognormal & 2.19 & 0.94 & 2,204,465 & 170\\

 & Travel - Transit & Lognormal & 2.04 & 0.67 & 2,618,247 & 168\\

\multirow[t]{-11}{*}{\raggedleft\arraybackslash 2010} & Work or school & Lognormal & 2.15 & 0.77 & 9,689,285 & 623\\
\cmidrule{1-7}
 & Business & Lognormal & 2.41 & 0.66 & 103,625 & 9\\

 & Cultural venues & Gamma & 4.21 & 0.33 & 690,105 & 50\\

 & Elsewhere & Lognormal & 2.61 & 0.78 & 1,856,985 & 116\\

 & Grocery store & Lognormal & 2.45 & 0.68 & 4,237,453 & 362\\

 & Health clinic & Lognormal & 2.42 & 0.70 & 337,453 & 30\\

 & Home & Lognormal & 2.56 & 0.74 & 18,796,315 & 1,310\\

 & Neighbourhood & Lognormal & 2.39 & 0.77 & 1,010,568 & 54\\

 & Other's home & Lognormal & 2.45 & 0.80 & 2,548,532 & 199\\

 & Outdoors & Lognormal & 2.58 & 0.75 & 1,498,086 & 84\\

 & Place of worship & Gamma & 5.21 & 0.27 & 364,366 & 26\\

 & Restaurant & Lognormal & 2.35 & 0.77 & 4,085,094 & 264\\

 & Sport area & Lognormal & 2.54 & 0.75 & 1,389,655 & 98\\

 & Travel - Motorized & Lognormal & 2.36 & 0.71 & 1,167,715 & 98\\

 & Travel - Transit & Lognormal & 2.25 & 0.59 & 2,119,106 & 130\\

\multirow[t]{-15}{*}{\raggedleft\arraybackslash 2015} & Work or school & Lognormal & 2.49 & 0.64 & 7,637,480 & 480\\
\cmidrule{1-7}
 & Business & Uniform & 0.00 & 20.97 & 56,268 & 5\\

 & Cultural venues & Gamma & 3.84 & 0.21 & 262,287 & 15\\

 & Elsewhere & Lognormal & 2.39 & 0.80 & 1,412,169 & 58\\

 & Grocery store & Lognormal & 2.74 & 0.63 & 4,906,922 & 208\\

 & Health clinic & Lognormal & 2.39 & 0.99 & 329,018 & 13\\

 & Home & Lognormal & 2.79 & 0.68 & 17,733,996 & 674\\

 & Neighbourhood & Lognormal & 2.28 & 0.88 & 963,957 & 33\\

 & Other's home & Lognormal & 2.23 & 0.80 & 1,466,294 & 87\\

 & Outdoors & Lognormal & 2.28 & 0.78 & 1,329,898 & 44\\

 & Place of worship & Uniform & 0.00 & 33.68 & 160,837 & 11\\

 & Restaurant & Lognormal & 2.35 & 0.70 & 2,266,291 & 111\\

 & Sport area & Lognormal & 2.62 & 0.45 & 656,375 & 41\\

 & Travel - Motorized & Lognormal & 2.51 & 0.68 & 1,173,687 & 64\\

 & Travel - Transit & Lognormal & 2.49 & 0.63 & 1,255,647 & 41\\

\multirow[t]{-15}{*}{\raggedleft\arraybackslash 2022} & Work or school & Lognormal & 2.60 & 0.63 & 8,135,455 & 222\\
\bottomrule
\end{tabular}
\begin{tablenotes}
\item \textit{Note: } 
\item For each probability distribution, Parameter 1 and Parameter 2 correspond to their standard defining parameters: the mean and standard deviation (for lognormal and normal), the rate and shape (for gamma), and the minimum and maximum bounds (for uniform). The AIC represents the Akaike Information Criterion used to assess model fit.
\end{tablenotes}
\end{threeparttable}}
\end{table}

\renewcommand{\thetable}{A.3}

\begin{table}
\centering
\caption{\label{tab:selected-functions}\label{tab:cycling-functions-tab}Impedance functions and AIC for cycling trips.}
\centering
\resizebox{\ifdim\width>\linewidth\linewidth\else\width\fi}{!}{
\fontsize{8}{10}\selectfont
\begin{threeparttable}
\begin{tabular}[t]{rllrrll}
\toprule
\multicolumn{1}{c}{\textbf{Year}} & \multicolumn{1}{c}{\textbf{Destination}} & \multicolumn{1}{c}{\textbf{Impedance function}} & \multicolumn{1}{c}{\textbf{Parameter 1*}} & \multicolumn{1}{c}{\textbf{Parameter 2*}} & \multicolumn{1}{c}{\textbf{AIC}} & \multicolumn{1}{c}{\textbf{Count}}\\
\midrule
 & Elsewhere & Lognormal & 2.40 & 0.74 & 886,149 & 35\\

 & Home & Lognormal & 2.74 & 0.84 & 1,779,272 & 66\\

 & Other's home & Lognormal & 2.55 & 0.83 & 380,492 & 12\\

 & Travel - Motorized & Lognormal & 2.50 & 1.10 & 51,578 & 2\\

\multirow[t]{-5}{*}{\raggedleft\arraybackslash 1992} & Work or school & Gamma & 3.05 & 0.17 & 443,095 & 20\\
\cmidrule{1-7}
 & Elsewhere & Lognormal & 2.84 & 0.74 & 839,485 & 35\\

 & Home & Gamma & 1.82 & 0.07 & 1,403,777 & 57\\

 & Other's home & Gamma & 1.15 & 0.06 & 216,650 & 11\\

 & Travel - Motorized & Lognormal & 2.49 & 0.87 & 71,466 & 2\\

\multirow[t]{-5}{*}{\raggedleft\arraybackslash 1998} & Work or school & Gamma & 3.18 & 0.09 & 492,595 & 20\\
\cmidrule{1-7}
 & Cultural venues & Uniform & 0.00 & 15.13 & 6,355 & 2\\

 & Elsewhere & Gamma & 2.04 & 0.12 & 309,694 & 28\\

 & Grocery store & Gamma & 2.34 & 0.17 & 372,629 & 31\\

 & Home & Gamma & 1.44 & 0.07 & 2,015,206 & 159\\

 & Other's home & Gamma & 1.78 & 0.14 & 323,185 & 29\\

 & Outdoors & Gamma & 2.65 & 0.15 & 230,078 & 18\\

 & Restaurant & Gamma & 3.34 & 0.21 & 113,033 & 11\\

 & Travel - Motorized & Lognormal & 3.39 & 0.45 & 53,641 & 3\\

\multirow[t]{-9}{*}{\raggedleft\arraybackslash 2005} & Work or school & Lognormal & 2.91 & 0.71 & 908,811 & 66\\
\cmidrule{1-7}
 & Cultural venues & Uniform & 0.00 & 32.58 & 38,938 & 3\\

 & Elsewhere & Lognormal & 2.30 & 0.36 & 168,783 & 14\\

 & Grocery store & Gamma & 2.49 & 0.15 & 392,826 & 22\\

 & Home & Lognormal & 2.59 & 0.78 & 2,164,764 & 116\\

 & Other's home & Lognormal & 2.40 & 0.63 & 338,777 & 19\\

 & Outdoors & Lognormal & 1.91 & 0.61 & 116,298 & 9\\

 & Restaurant & Uniform & 0.00 & 17.49 & 35,370 & 3\\

 & Travel - Motorized & Gamma & 1.53 & 0.09 & 77,569 & 5\\

\multirow[t]{-9}{*}{\raggedleft\arraybackslash 2010} & Work or school & Lognormal & 2.66 & 0.76 & 1,332,309 & 55\\
\cmidrule{1-7}
 & Cultural venues & Lognormal & 2.71 & 0.00 & -Inf & 2\\

 & Elsewhere & Gamma & 3.59 & 0.13 & 120,388 & 9\\

 & Grocery store & Lognormal & 2.99 & 0.82 & 252,582 & 16\\

 & Health clinic & Lognormal & 2.93 & 0.86 & 80,810 & 4\\

 & Home & Lognormal & 3.09 & 0.66 & 2,008,783 & 112\\

 & Neighbourhood & Uniform & 0.00 & 48.55 & 49,924 & 3\\

 & Other's home & Lognormal & 2.52 & 0.44 & 140,210 & 12\\

 & Outdoors & Uniform & 0.00 & 35.03 & 31,463 & 3\\

 & Restaurant & Lognormal & 3.11 & 0.60 & 115,406 & 9\\

 & Sport area & Uniform & 0.00 & 17.47 & 32,969 & 6\\

 & Travel - Motorized & Uniform & 0.00 & 44.07 & 54,215 & 3\\

 & Travel - Transit & Uniform & 0.00 & 22.33 & 14,495 & 2\\

\multirow[t]{-13}{*}{\raggedleft\arraybackslash 2015} & Work or school & Lognormal & 3.03 & 0.41 & 1,173,460 & 65\\
\cmidrule{1-7}
 & Grocery store & Normal & 45.50 & 21.58 & 634,305 & 10\\

 & Health clinic & Uniform & 0.00 & 61.98 & 123,138 & 3\\

 & Home & Gamma & 4.04 & 0.15 & 1,798,770 & 56\\

 & Outdoors & Lognormal & 3.44 & 0.11 & 94,804 & 2\\

 & Restaurant & Normal & 26.29 & 5.83 & 100,585 & 3\\

 & Sport area & Normal & 29.89 & 17.05 & 124,824 & 5\\

 & Travel - Motorized & Uniform & 0.00 & 37.38 & 11,980 & 2\\

 & Travel - Transit & Uniform & 0.00 & 16.83 & 12,569 & 2\\

\multirow[t]{-9}{*}{\raggedleft\arraybackslash 2022} & Work or school & Gamma & 5.87 & 0.24 & 832,953 & 37\\
\bottomrule
\end{tabular}
\begin{tablenotes}
\item \textit{Note: } 
\item For each probability distribution, Parameter 1 and Parameter 2 correspond to their standard defining parameters: the mean and standard deviation (for lognormal and normal), the rate and shape (for gamma), and the minimum and maximum bounds (for uniform). The AIC represents the Akaike Information Criterion used to assess model fit.
\end{tablenotes}
\end{threeparttable}}
\end{table}

\bibliography{mybibfile2}


\end{document}
